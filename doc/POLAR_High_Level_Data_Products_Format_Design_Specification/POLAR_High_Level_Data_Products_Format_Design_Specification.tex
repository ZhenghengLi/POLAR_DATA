\documentclass[a4paper, 12pt, onecolumn]{article}
\usepackage{graphicx}
\usepackage{amsmath}
\usepackage{fancyhdr}
\usepackage[colorlinks]{hyperref}
\usepackage{longtable}
\usepackage{fancyvrb}
\begin{document}
% ==== title page =====================
\begin{titlepage}
  \vspace*{5cm}
  \begin{center}
    \huge \bf POLAR High Level Data Products Format Design Specification
  \end{center}
  \vspace*{6cm}
  \begin{center}
    \large
    \begin{tabular}{ | l | l | l | }
      \hline
      & \bf Name & \bf Date \\ \hline
      Prepared by & Zhengheng Li & May 20, 2016 \\ \hline
    \end{tabular}
  \end{center}
\end{titlepage}

% ==== header and footer
\pagestyle{fancy}
\lhead{}
\chead{\small POLAR High Level Data Products Format Design Specification}
\rhead{}
\lfoot{\today}
\cfoot{}
\rfoot{\thepage}
\renewcommand{\headrulewidth}{0.4pt}
\renewcommand{\footrulewidth}{0.1pt}

% ==== table of conents ===============
\pagenumbering{Roman}
\tableofcontents
\newpage
\pagenumbering{arabic}

% ==== begin main contents ================
\section{Introduction}
This chapter contains an introduction to the document ``POLAR High Level Data Products Format Design Specification''

\subsection{Purpose of the document}
Three core pre-processing programs of POLAR SCI and HK raw data have been finished. 
They are \texttt{SCI\_Decode}, \texttt{HK\_Decode} and \texttt{Time\_Calculate}.
For raw data products from POAC, please see the document\cite{POAC-data-products}.
\texttt{SCI\_Decode} is to directly decode 0B level POLAR SCI raw data from POAC, and do time sync at the same time. 
\texttt{HK\_Decode} is to directly decode 0B level POLAR HK raw data from POAC, and do some physical value converting work.
\texttt{Time\_Calculate} is to calculate the absolute GPS time of each event in SCI decoded data using the GPS and timestamp sync information in HK decoded data.
These three programs are tested by lots of ground data and work well. One important thing is the format or data structure of the output data files.
Everyone who uses these programs should know the format and the way of data organization.
This document is mainly to clarify the data structure of decoded data produced by the three pre-processing programs.

\subsection{Levels of data products}
POLAR data products has several different levels. 1M level data is the directly decoded data produced by \texttt{SCI\_Decode} or \texttt{HK\_Decode}.
It should keep all information in 0B level raw data, and add some auxiliary data which is helpful for data monitor and data analysis later. 
The level of SCI data after absolute GPS time of each event is calculated and added by \texttt{Time\_Calculate} is 1P.
1M and 1P level SCI data have almost the same data structure except for absolute GPS time added.
HK data does not have 1P level, because 1M level HK data already have absolute GPS time.

One raw data file from POAC could be very big, because it may contain a day of data.
The time span of one orbit is about 90 minutes, so it could be convenient to split the data by orbit.
The data structure of orbit splitted data should be the same as the data that is not splitted.
So, data monitor and data analysis software can directly process the data after and before splitted without any change.
The level of orbit splitted data is 1R.

This document will give a clear clarification of data structure of 1M and 1P level SCI decoded data, 1M level HK decoded data.
SCI data of one event include one trigger packet and one or more module packets.
It is important to understand the data organization of event data in the output ROOT file.

\section{Usage of the three programs}
Before introducing the data products format, this chapter gives a brief introduction to how to use the three core pre-processing programs.

\subsection{Usage of \texttt{SCI\_Decode} and \texttt{HK\_Decode}}
The way of using the two decoding programs \texttt{SCI\_Decode} and \texttt{HK\_Decode} are the same, we can run one of them without any command line parameters to see the help information.

Help information of \texttt{SCI\_Decode} is as following:
\begin{Verbatim}[fontsize=\scriptsize, frame=single]
> SCI_Decode
Usage:
  SCI_Decode [-l <listfile.txt>] [<POL_SCI_data_001.dat> <POL_SCI_data_002.dat> ...]
             [-o <POL_SCI_decoded_data.root>] [-g <POL_SCI_decoding_error.log>]

Options:
  -l <listfile.txt>                text file that contains raw data file list
  -o <decoded_data.root>           root file that stores decoded data
  -g <decoding_error.log>          text file that records decoding error log info

  --version                        print version and author information
\end{Verbatim}

And help information of \texttt{HK\_Decode} is as following:
\begin{Verbatim}[fontsize=\scriptsize, frame=single]
> HK_Decode
Usage:
  HK_Decode [-l <listfile.txt>] [<POL_HK_data_001.dat> <POL_HK_data_002.dat> ...]
            [-o <POL_HK_decoded_data.root>] [-g <POL_HK_decoding_error.log>]

Options:
  -l <listfile.txt>                text file that contains raw data file list
  -o <decoded_data.root>           root file that stores decoded data
  -g <decoding_error.log>          text file that records decoding error log info

  --version                        print version and author information
\end{Verbatim}

There are two ways to input raw data files.

The first way is directly to use command line parameters without options to give file names as following:

\begin{Verbatim}[fontsize=\scriptsize, frame=single]
> SCI_Decode POL_SCI_data_20160517_154345_001.dat POL_SCI_data_20160517_154345_002.dat ...
\end{Verbatim}

\texttt{SCI\_Decode} will scan the designated raw data files one by one from left to right and generate only one decoded ROOT file.
The default name of the output file is \texttt{POL\_SCI\_decoded\_data.root} for \texttt{SCI\_Decode} if it is not specified by option \texttt{-o}.

The second way is to use a text file which contains all the file names line by line. And use option \texttt{-l} to input the raw data files. Just as following:

\begin{Verbatim}[fontsize=\scriptsize, frame=single]
> cat listfile.txt
path/to/POL_SCI_data_20160517_154345_001.dat
path/to/POL_SCI_data_20160517_154345_002.dat
path/to/POL_SCI_data_20160517_154345_003.dat
...
> SCI_Decode -l listfile.txt
\end{Verbatim}

Options \texttt{-o} and \texttt{-g} are optional. We can use option \texttt{-o} to specify the name of output decoded file.
If option \texttt{-g} is used, \texttt{SCI\_Decode} and \texttt{HK\_Decode} will record some log information into a text file,
including the raw data of bad packets.

After a run of \texttt{SCI\_Decode} or \texttt{HK\_Decode} finished, some counter information will be printed out,
including count of total frames and packets, count of CRC error, count and percentage of packets lost, percentage of time aligned event packets, etc..
Such counter information can give some indications of quality of the raw data.

Screen output of \texttt{SCI\_Decode} is as following:
\begin{Verbatim}[fontsize=\tiny, frame=single]
POL_SCI_data_20160517_154345_001.dat
POL_SCI_data_20160517_154345_002.dat
POL_SCI_data_20160517_154345_003.dat
===========================================================================================================
total frame count:          783485              total packet count:         17786003  
frame invalid count:        0                   - trigger packet count:     8090369   
frame invalid percent:      0.00%               - event packet count:       9695515   
frame crc error count:      0                   packet invalid count:       65        
frame crc error percent:    0.00%               packet invalid percent:     0.00%     
frame interruption count:   0                   packet crc error count:     633       
frame start error count:    0                   packet crc error percent:   0.00%     
total timestamp 0 count:    0                   packet too short count:     291                 
-----------------------------------------------------------------------------------------------------------
  ct  mod  >   ped_trig  ped_event   ped_lost  percent    |   noped_trig  noped_event   noped_lost  percent
   1  405  >        766        766          0    0.00%    |       261973       261973            0    0.00%
   2  639  >        766        766          0    0.00%    |       340300       340300            0    0.00%
   3  415  >        765        765          0    0.00%    |       359015       359014            1    0.00%
   4  522  >        758        758          0    0.00%    |       361436       361436            0    0.00%
   5  424  >        763        763          0    0.00%    |       322721       322721            0    0.00%
   6  640  >        763        763          0    0.00%    |       317664       317663            1    0.00%
   7  408  >        760        760          0    0.00%    |       406439       406439            0    0.00%
   8  638  >        757        757          0    0.00%    |       448543       448543            0    0.00%
   9  441  >        758        758          0    0.00%    |       471523       471523            0    0.00%
  10  631  >        758        758          0    0.00%    |       418859       418859            0    0.00%
  11  411  >        769        769          0    0.00%    |       305021       305021            0    0.00%
  12  505  >        757        756          1    0.13%    |       426402       426403           -1   -0.00%
  13  503  >        759        759          0    0.00%    |       495925       495925            0    0.00%
  14  509  >        742        742          0    0.00%    |       519941       519941            0    0.00%
  15  410  >        762        762          0    0.00%    |       420677       420677            0    0.00%
  16  507  >        769        769          0    0.00%    |       321857       321857            0    0.00%
  17  402  >        758        758          0    0.00%    |       392200       392200            0    0.00%
  18  602  >        754        754          0    0.00%    |       506862       506861            1    0.00%
  19  414  >        765        765          0    0.00%    |       482388       482388            0    0.00%
  20  524  >        747        746          1    0.13%    |       437999       437999            0    0.00%
  21  423  >        766        766          0    0.00%    |       246196       246194            2    0.00%
  22  601  >        761        761          0    0.00%    |       365308       365308            0    0.00%
  23  406  >        770        767          3    0.39%    |       326266       325397          869    0.27%
  24  520  >        771        771          0    0.00%    |       402897       402897            0    0.00%
  25  413  >        768        768          0    0.00%    |       317960       317960            0    0.00%
-----------------------------------------------------------------------------------------------------------
trigg expected sum: 9695404        noped_trigger:   8089584             ped_trigger:     785                 
event received sum: 9694526        noped_event_sum: 9675499             sec_ped_trigger: 636                 
total lost percent: 0.01%          mean event rate: 12719 cnts/sec      np_evts per sec: 15213 pkts/sec      
transmission rate:  19.96 Mbps     aligned sum:     9675497             aligned percent: 100.00%             
===========================================================================================================
\end{Verbatim}

Screen output of \texttt{HK\_Decode} is as following:

\begin{Verbatim}[fontsize=\tiny, frame=single]
POL_HK_data_20160517_154345_001.dat
=============================================================================================
total frame count:          12564               total obox packet count:    6282                
frame valid count:          12564               obox valid count:           6281                
frame invalid count:        0                   obox invalid count:         1                   
frame crc passed:           12564               obox crc passed:            6281                
frame crc error count:      0                   obox crc error count:       0                   
frame interruption count:   0                   
=============================================================================================
\end{Verbatim}

\subsection{Usage of \texttt{Time\_Calculate}}
\texttt{Time\_Calculate} is used to calculate and add the absolute GPS time of each event in decoded SCI data.
It can work only when the GPS time in HK data is valid. We can also run this program without any command line parameters to see the help information.

Help information of \texttt{Time\_Calculate} is as following:
\begin{Verbatim}[fontsize=\tiny, frame=single]
Usage:
  Time_Calculate <POL_SCI_decoded_data.root> -k <POL_HK_decoded_data.root>
                 [-o <POL_SCI_decoded_data_time.root>] [-g <POL_SCI_time_error.log>]

Options:
  -k <hk_decoded_data.root>        root file that stores hk decoded data
  -o <sci_decoded_data.root>       root file that stores sci decoded data after absolute time is added
  -g <time_error.log>              text file that records time calculating error log info

  --version                        print version and author information
\end{Verbatim}

It is very straightforward. Just use option \texttt{-k} to designate the file name of decoded HK data.
Options \texttt{-o} and \texttt{-g} are also optional. Option \texttt{-o} is used to specify the file name of the output ROOT file that stores the SCI data after
absolute GPS time is added. If option \texttt{-o} is not used, the default file name is \texttt{POL\_SCI\_decoded\_data\_time.root}.
When option \texttt{-g} is used, this program will record some error log information into a text file.

Screen output of \texttt{Time\_Calculate} is as following:

\begin{Verbatim}[fontsize=\scriptsize, frame=single]
Copying physical modules data ...
[ ################################################# DONE ] 
Calculating time and copying physical trigger data ...
[ ################################################# DONE ] 
Copying pedestal modules data ...
[ ################################################# DONE ] 
Calculating time and copying pedestal trigger data ...
[ ################################################# DONE ] 
================================================================================
phy_error_count: 0 / 8089584         ped_error_count: 0 / 785             
================================================================================
\end{Verbatim}

Absolute GPS time is only added into trigger packets, and all of other data is just copied.

\section{Data Structure of ROOT files}


% ==== references =======================
\begin{thebibliography}{9}
  \bibitem{POAC-data-products} POLAR\_space\_data\_from\_GESSA/POAC data products.pdf
\end{thebibliography}

\end{document}
